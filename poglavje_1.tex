\inputencoding{utf8}
\chapter[Testiranje zmogljivosti Amazon EC2 platforme (P. Matičič, J. Pelicon, B. Rojc)]{Testiranje zmogljivosti Amazon EC2 platforme}

\pagestyle{fancy}
\fancyhf{}
\fancyhead[LE,RO]{\thepage}
\fancyhead[RE,LO]{\leftmark}

\huge Peter Matičič, Jan Pelicon, Blaž Rojc
\normalsize
\bigskip

\section{Opis problema}

Z dneva v dan proizvedemo čedalje več slik.
Predstavljajo znaten delež podatkov, shranjenih v raznih storitvah v oblaku.
Ampak ko želimo najti določen predmet ali osebo, ki smo jo slikali, je ročno brskanje po digitalnih zbirkah zamudno.

S tem problemom v mislih bomo stestirali oblačno platformo Amazon EC2.
Ustvarili bomo enostavno storitev, ki bo uporabniku omogočala iskanje vzorcev v večjem naboru slik, shranjenih v oblaku.
Poglobili se bomo v zahtevnost uporabe za programerja, fleksibilnost pri programiranju in morebitnem prenašanju storitve na druge platforme,
	odzivnost in izkušnjo za uporabnika ter zmogljivost in skalabilnost virov na platformi.

\section{Izbira tehnologij}

\section{Definicija bremena storitve}

\section{Definicija metrik in orodij za meritve}

\section{Rezultati meritev}

\section{Plan dela}

Snovanje ogrodja storitve, določitev bremen, metrik, orodij, \ldots

\inputencoding{cp1250}